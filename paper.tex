\documentclass[a4paper, twocolumn]{article}
\usepackage[utf8]{inputenc}
\usepackage{blindtext}
\usepackage{cite}
\usepackage{comment}
\usepackage[a4paper,left=2cm,right=2cm,top=2.5cm,bottom=2.5cm]{geometry}

\title{\Large{Research Seminar Software Science} \\
\huge{Monads}}
\author{Brigel Pineti}
\date{April 2018}

\begin{document}

\maketitle

\section{Introduction}

The code snippets presented in this paper are implemented with Haskell. 

\section{Haskell}

In this section are reflected two main features concerning the above mentioned functional language. Taking into consideration that not every reader might be familiar with the language, I found it necessary to briefly illustrate the laziness and purity of Haskell.  

\subsection{Lazy Evaluation}

Haskell is a functional language with non-strict semantics. What does this mean? An expression is evaluated from the outside in. For instance, the evaluation of \textit{$(a + (b * c))$} will firstly reduce the \textit{$+$} operator and then the inner bracket \textit{$(a * b)$}. \\
Lazy evaluation is just an implementation approach for a non-strict language. An appropriate definition for laziness can be represented by the term \textit{call-by-need}. Consequently, the evaluation of an expression is delayed until its result is actually needed from another computation. Therefore, the parameters of a function are evaluated only when they are needed and not a moment before. 
\end{document}
