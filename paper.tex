\documentclass[a4paper, onecolumn]{article}
\usepackage[utf8]{inputenc}
\usepackage{blindtext}
\usepackage{cite}
\usepackage{comment}
\usepackage{color}
\usepackage[a4paper,left=3cm,right=3cm,top=4cm,bottom=4cm]{geometry}

\title{\Large{Research Seminar Software Science} \\
\huge{Simulation of monadic effects in Haskell}}
\author{Brigel Pineti}
\date{April 2018}

\begin{document}

\maketitle

\section*{Abstract}

\section{Introduction}

Pure function languages like Haskell rule out side effects. Why? Firstly, to allow unrestricted application of program transformation and equation logic. Secondly, being a non-strict language, Haskell's order of side effects is undefined. The aim of this paper is to show how monads can be used to exploit impure effects into a pure language. The code snippets attached to this paper are implemented in Haskell. Section 2 will provide an overview on the characteristics of the language being used. Section 3 is using the Maybe Monad as an illustrative example to get the reader's feet wet with the complex concept of a Monadic value. The application of monads is demonstrated with three case studies.

\section{Haskell}

In this section are reflected two main features concerning the above mentioned functional language. Taking into consideration that not every reader might be familiar with the properties of the language, I found it necessary to briefly illustrate the laziness and purity of Haskell.  

\subsection{Lazy Evaluation}

Haskell is a functional language with non-strict semantics. What does this mean? An expression is evaluated from the outside in. For instance, the evaluation of \textit{$(a + (b * c))$} will firstly reduce the \textit{$+$} operator and then the inner bracket \textit{$(a * b)$}. \\
Lazy evaluation is just an implementation approach for a non-strict language. An appropriate definition for laziness can be represented by the term \textit{call-by-need}. Consequently, the evaluation of an expression is delayed until its result is actually needed from another computation. Therefore, the parameters of a function are evaluated only when they are needed and not a moment before. An advantage of laziness 

\subsection{Absence of Side Effects}

A straightforward consequence of laziness is the evaluation order of a Haskell program. Being demand-driven, it becomes impossible for a function call to produce any side effects. Therefore, Haskell is considered to be a \textit{pure language}. For example, a function of type $Char \rightarrow Char$ will neither write nor read any mutable variables. Nevertheless, side effects are without any doubt very convenient. Haskell's restricted side effects resulted in a sluggish input/output. For this reason, monadic I/O was created; thus by leading to a functional language with side effects. 
\section{Monads}
Monads originate from a section of mathematics called category theory. Luckily for us, it is not required to have prior knowledge in category theory in order to comprehend the application of monads in Haskell. 
Nevertheless, there exists a certain complexity when it comes to grasping to concept of a monad. From history, monads were initially adopted in Haskell to produce input and output operations. However, the capabilities of monads were not bound to just I/O. Additionally, monads can support multiple operations such as state, non-determinism, continuations, exceptions etc. \\ \\ 
The default monad is composed by three essential components: 
\begin{itemize}
    \item a type constructor
    \item a function return
    \item a bind operator ($\gg$=)
\end{itemize}
The \textit{return} function and the bind operator ($\gg$=) have the following types: 

\begin{center}
   return   :: a $\rightarrow$ m a \\
   ($\gg$=) :: m a $\rightarrow$ (a $\rightarrow$ m b) $\rightarrow$ m b 
   \end{center}
In order for the above function definitions to hold, the monad laws need to be satisfied. There exist three monadic laws: 

\begin{itemize}
    \item right unit \hspace{20mm} m $\gg$= return \textcolor{red}{=} m 
    \item left unit \hspace{20mm} return x $\gg$= f \textcolor{red}{=} f x   
    \item associative \hspace{15mm} (m $\gg$= f) $\gg$= g \textcolor{red}{=} m $\gg$= (\textbackslash x $\rightarrow$ f x $\gg$= g) 
\end{itemize}
According to the laws of right and left unit, the return serves as a collector of a value where no computation occurs. As the type of \textit{return} suggests, it takes a value of type \textit{a}, puts it in a special context \textit{m} and returns as a result \textit{m a}.
The Maybe monad has been selected to provide a concrete insight on how a monad functions.   


\end{document}
