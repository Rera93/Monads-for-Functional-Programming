\documentclass{article}
    \usepackage[utf8]{inputenc}
    \usepackage[a4paper,left=1.5cm,right=1.5cm,top=1.5cm,bottom=3cm]{geometry}
    
    \title{\Large{Peer Review} \\
    \huge\textbf{{A Modern Look at Introductory Programming Languages}}}
    \author{Brigel Pineti}
    \date{May 2018}
    
    \begin{document}
    
    \maketitle
    
    \section*{}
    \textbf{1. } In order to keep the reader in track with what is happening throughout the paper, the writers may clearly indicate in the beginning of each section, the purpose of every subsection. The current version of the paper contains only one section (\textit{Criteria}) besides the usual \textit{Introduction} section. The reader is being implicitly informed one the content of the \textit{Criteria} section. Sentence after sentence the content of the section is being revealed. However, it is still a bit vague and I would suggest the writers on each sentence to explicitly state the subsection number and a brief description on the intent of that subsection. I believe, the same approach should be applied to each section, including the \textit{Introduction}. Nevertheless, this cannot be achieved yet since there are supplementary sections to be added.  
    
    \section*{}
    \textbf{2. }  Up until now, the work seems to be produced by the authors themselves. Additionally, sentences are being correctly referenced. For example, the sentence \textit{``In a more recent study Nanz and Furia [5] compare various programming languages based on their performance on simple algorithmic tasks from the Rosetta code repository"} is stating a fact and if the reader is doubting on what is being stated, he/she may check the references for further proof.
    
    \section*{}
    \textbf{3. } At the moment, no evaluation has been provided. Although, I believe it is a little early to ask this question.
    
    \section*{}
    \textbf{4 .} Layout and formatting is acceptable by me. I like that the paper is constructed as a two-column paper. In my opinion, it makes it easier for the reader to follow the material.
    
    \section*{}
    \textbf{5 .} The English language being used is good enough. Nonetheless, I had one suggestion on the writing style. In the introduction of the \textit{Criteria} section, the word criteria is being used 6 times in just 7 sentences. I doesn't sound nice to hear in your head while reading one word repeated so often. Thus, I would recommend avoiding repetitions. In seven sentences, the word criteria could be used at most 2 times. In order to achieve this, synonyms could be used or sentences could be constructed in such a way that the usage of the word would be minimized. 
    
    \section*{}
    \textbf{6 .} As a final advice, I would suggest to the writers to provide more sections on the next intermediate representation. Viewing it from the reader's perspective, the paper does not provide a concrete idea on where is headed.  
    \end{document}
    